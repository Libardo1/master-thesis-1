\chapter*{Въведение}\addcontentsline{toc}{chapter}{Въведение}\chaptermark{Въведение}

	Играете ли \emph{Angry Birds\footnote{\url{http://www.angrybirds.com/}}}? Повече от 260 милиона\footnote{\url{http://mashable.com/2013/01/11/rovio-angry-birds-260-million/}} потребители го правят редовно! Може би игрите не ви допадат? Може спокойно да изберете някое от другите 900,000$+$ приложения за мобилни устройства. Кои от тях са подходящи?
	
	Апликации и игри за мобилни устройства варират спрямо категории, целеви групи и др. Над един милиард\footnote{\url{http://cuttinglet.com/world-smartphone-population-crosses-1-billion/}} от земното население има ежедневен достъп до тях. Децата се забавляват с увлекателни игри, докато по-възрастните използват телефоните си за банкиране, запазване на самолетни билети, планиране на екскурзии и др.
	
	От другата страна на този феномен са разработчиците на мобилни приложения. Те се опитват да слеят границите между истинския свят и виртуалната реалност. Явен пример за това е \emph{Ingress\footnote{\url{http://www.ingress.com/}}}. Способността да имаме различна, виртуална идентичност, кара милиони потребители да се състезават, помагат, следят и обсъждат с непознати и приятели.
	
	Голямото разнообразие на мобилни приложения поставя сериозен въпрос - Как да намеря най-доброто приложение за мен? В ерата на \emph{Google} търсенето, този въпрос може да изглежда безмислен, но това съвсем не е така. Текстът на едно приложение е само неговото заглавие и описание. Той може да е неточен и непълен. Още повече, функционалността и интерфейса може да са неправилно създадени.
	
	\ac{RS} предоставят разрешение на този проблем. Те създават предложения, които използват данни от други потребители и техните оценки. Фактори като общ рейтинг, брой сваляния, категория и класиране, също се взимат под предвид.

	\subsubsection{Цел}
	
		Целта на дипломната работа е да изгради система за препоръки на мобилни приложения за \emph{Android}.

	\subsubsection{Изисквания}

		\begin{itemize}
			\item Лесна за употреба.
			\item Предоставя на добри препоръки.
			\item Ниска цена за поддръжка.
			\item Лесна за разширение и подобрение.
		\end{itemize}

	\subsubsection{Постигане на целта}

		За постигане на целите на дипломната работа, спрямо поставените изисквания се разглеждат:
		
		\begin{itemize}
			\item Модерни технологични решения в подобни ситуации.
			\item Проучване на съществуващи системи.
			\item Създаване на система за препоръки на мобилни приложения.
			\item Получените резултати се анализират, за да се получи частично или пълно решение на проблема.
		\end{itemize}

	\subsubsection{Целеви групи}
	
		Потребители, които желаят по-добри приложения, максимално приближаващи се до техните вкусове и изисквания. Важно за тях е бързото намиране и инсталиране на голям брой апликации, които да ползват за по-дълго време.
	
	\subsubsection{Желан резултат}
	
		Дипломната работа може да се сметне за успешна, когато бъде създаден прототип на система, който да позволява 
		предоставянето на препоръки за мобилни приложения. Системата трябва да бъде лесна за промяна и разширяване,
		така че да предоставя начини за добавяне на нови източници на информация и създаване на клиенти.
	
	\subsubsection{Задачи}
		
		За постигане на желания резултат е необходимо следните задачи да бъдат изпълнени:	
	
		\begin{itemize}
			\item Проучване на видове \ac{RS}
			\item Съществуващи примери
			\item Избор на подходящи технологии
			\item Моделиране на отделните компоненти
			\item Създаване на тестове
			\item Създаване на системата
			\item Провеждане на тестове върху системата
			\item Превеждане в употреба
			\item Препоръки за подобрения
		\end{itemize}