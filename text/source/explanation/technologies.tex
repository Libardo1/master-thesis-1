\section{Използвани технологии}
	\subsection{Сървърен модул}
		
		Сървърната част от системата е необходимо да бъде в постоянна готовност за обработка на нови данни, извличане на такива и предоставянето им
		на някой от клиентите. За извършване на своята функция, всяка от посочените операции, комуникира с услуга, която предоставя запазване и извличане на информация. Тя е критична точка в системата, тъй като всеки друг сървърен модул зависи от нейното правилно действие.
	
		\subsubsection{MongoDB}
		
			MongoDB\footnote{\url{http://www.mongodb.org/}}	е документно-ориентирана база от данни, която не използва SQL\footnote{\url{http://www.w3schools.com/SQl/default.asp}}. Тя предоставя скалируемост, копиране(replication), индекси, извличане на информация и др.
			MongoDB запазва данни в JSON\footnote{\url{http://json.org/}} подобен формат, като документи, вместо таблици.
			
			Предоставени са множество библиотеки за комуникация с данните за популярните програмни езици\footnote{\url{http://docs.mongodb.org/ecosystem/drivers/}}. Възможна е и връзка чрез REST и HTTP\footnote{\url{http://www.mongodb.org/display/DOCS/Http+Interface}}.
			Системата предоставя бързо изграждане на прототипи и реални приложения като нуждата от създаване на таблици и схеми се избягва напълно.
			Негативен фактор е липсата на ясно изразен резултат от това дали даден запис е бил съхранен успешно\cite{Sirer}
			
			MongoDB се използва в множество бизнес проекти и продължава да трупа популярност. Предоставя се и много добра интеграция от водещи доставчици на облъчни услуги: Amazon Web Services, Microsoft Windows Azure, Rackspace Cloud, Red Hat OpenShift and VMware Cloud Foundry и други\cite{Marketwire}.
		
		\subsubsection{Python}
		
		\emph{Python\footnote{\url{http://python.org/}}} е език за програмиране от високо ниво, който се стреми да предостави четим код. Предоставя начин за писане на по-малко редове, спрямо \emph{C}. Той поддържа множество парадигми, включително обектно-ориентиран, императивен и функционален стил. Код написан на \emph{Python} може да бъде изпълняван върху множество платформи. 
		
		Езикът предоставя множество допълнителни модули за работа с графични среди, математически операции, мрежи, уеб сървъри и други. Това богаство го прави много подходящ за почти всякакъв вид приложения. \emph{Python} е широко използван в научни и бизнес среди\footnote{\url{http://python.org/about/quotes/}}.
		
		\subsubsection{Scrapy}

		\emph{Scrapy\footnote{\url{http://scrapy.org/}}} е библиотека с отворен код от високо ниво за събиране на информация от уеб пространството. Използва се за извличане на структурирани данни и автоматично тестване.
		
		Библиотеката е бърза и лесна за работа, като предоставя и добра документирана. Предоставя паралелна обработка на данните, както и самите заявки към страниците. Основният метод за извличане на информация е чрез \emph{XPath} изрази.
		
		\subsubsection{mongoengine}

		Тази библиотека свързва \emph{MongoDB} и сървърната част на приложението. \emph{Mongoengine\footnote{\url{Mongoengine}}} предоставя обектно-ориентиран подход за операриране с \ac{DB}. Добре документирана е и предлага лесен начин за използване. \emph{Mongoengine} надгражда функционалност върху \emph{PyMongo}, който е драйвър за \emph{MongoDB}.
		
		\subsubsection{bottle}

		\emph{Bottle\footnote{\url{http://bottlepy.org/docs/dev/}}} е бърза, микро библиотека за изграждане на малки уеб приложения. Тя предоставя лесно извличане на параметри от заявка, интеграция с различни двигатели за изгледи, вграден сървър и възможност за използване на по-големи такива. \emph{Bottle} не зависи от нищо друго, освен стандартната библиотека на \emph{Python}. 
		
		Библиотеката е особено подходяща за създаване на \emph{REST}-пълни \ac{API}, по лесен и бърз начин. Приложение, изградено с \emph{bottle}, се тества лесно чрез вградени механизми.
		
	\subsection{Клиент}
		
		От страна на клиентската част, библиотеките, които се ползват са пряко зависими от платформите на които приложението трябва да се използва.	
		За разработката на добро \emph{Android} приложение, езикът, който предоставя най-големи възможности и библиотеки е \emph{Java}. Това, разбира се, налага някои ограничения.
		
		\subsubsection{Android SDK}

		\ac{SDK} на \emph{Android} предоставят голяма част от стандартната версия на \emph{Java} \ac{SDK}. Съществуват и много допълнения, като някои от тях предоставят възможности за:
		
		\begin{itemize}
			\item работа с графичната среда на \emph{Android}
			\item услуги
			\item бази от данни
			\item карти
			\item GPS
			\item Жироскоп
			\item връзка с интернет
		\end{itemize}
		
		Предоставя се и лесна интеграция с \emph{Eclipse\footnote{\url{http://www.eclipse.org}}}, \emph{IntelliJ\footnote{\url{http://www.jetbrains.com/idea/}}} и други среди за разработка. За процесът на пакетиране на приложението се използват \emph{Maven\footnote{\url{http://maven.apache.org/}}}, \emph{Gradle\footnote{\url{http://www.gradle.org/}}}, \emph{Ant\footnote{\url{http://ant.apache.org/}}}и др.
		
		Допълнителни функционалности се интегрират постоянно в по-новите версии. Извършва се тяхното частично добавяне в по-стари версии, чрез т.нар. \emph{Android support library}, която се поддържа от официалните разработчици на \emph{Android}.
		
		\subsubsection{Retrofit}
		
		\emph{Retrofit\footnote{\url{http://square.github.io/retrofit/}}} предоставя стриктно типизиран подход към \emph{REST} \ac{API} за \emph{Android}. С използване на библиотеката, част от работата на програмиста се спестява като се използва метапрограмиране и по-точно анотации в \emph{Java}.
		
		Библиотеката позволява изпращането на всички видове възможни заявки и добавяне на всякаква информация към тях. Поддържат се различни типове документи, като обработката на \emph{JSON} е лесна и ефективна, при получаване и изпращане. Възможно е прикачването на файлове и изпращане на форми.
		
		\subsubsection{RoboSpice}
		
		\emph{Retrofit} е много добра библиотека за извършване на заявки, но не предоставя начин за кеширане на резултати, определяне време за изпълнение на заявки, използване на много ядра, изпълнение в многонишкова среда и др.
		
		Всички свойства се предоставя от \emph{RoboSpice\footnote{\url{https://github.com/octo-online/robospice}}}. Библиотеката е специфично създадена за работа при мобилни устройства и предоставя защита от блокиране, показване на грешки и загуба на памет.
		
		Благодарение на модуларната си структура, \emph{RoboSpice} се интегрира лесно за работа с \emph{Retrofit} и я използва за извършване на самите заявки.
		
		Библиотеката предоставя механизъм за извличане на информация от наличната на устройството и спестява извършване на тежки мрежови операции. За това се използват различни начини за запазване на информация като файлове, \ac{DB} и др.