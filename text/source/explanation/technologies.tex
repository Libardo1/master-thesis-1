\section{Използвани технологии}
	\subsection{Сървърен модул}
		
		Сървърната част от системата е необходимо да бъде в постоянна готовност за обработка на нови данни, извличане на такива и предоставянето им
		на някой от клиентите. За извършване на своята функция, всяка от посочените операции, комуникира с услуга, която предоставя запазване и извличане на информация. Тя е критична точка в системата, тъй като всеки друг сървърен модул зависи от нейното правилно действие.
	
		\subsubsection{MongoDB}
		
			MongoDB\footnote{\url{http://www.mongodb.org/}}	е документно-ориентирана база от данни, която не използва SQL\footnote{\url{http://www.w3schools.com/SQl/default.asp}}. Тя предоставя скалируемост, копиране(replication), индекси, извличане на информация и др.
			MongoDB запазва данни в JSON\footnote{\url{http://json.org/}} подобен формат, като документи, вместо таблици.
			
			Предоставени са множество библиотеки за комуникация с данните за популярните програмни езици\footnote{\url{http://docs.mongodb.org/ecosystem/drivers/}}. Възможна е и връзка чрез REST и HTTP\footnote{\url{http://www.mongodb.org/display/DOCS/Http+Interface}}.
			Системата предоставя бързо изграждане на прототипи и реални приложения като нуждата от създаване на таблици и схеми се избягва напълно.
			Негативен фактор е липсата на ясно изразен резултат от това дали даден запис е бил съхранен успешно\cite{Sirer}
			
			MongoDB се използва в множество бизнес проекти и продължава да трупа популярност. Предоставя се и много добра интеграция от водещи доставчици на облъчни услуги: Amazon Web Services, Microsoft Windows Azure, Rackspace Cloud, Red Hat OpenShift and VMware Cloud Foundry и други\cite{Marketwire}.
		
		\subsubsection{Python}
		\subsubsection{scrapy}
		\subsubsection{mongoengine}
		\subsubsection{bottle}
	\subsection{Клиент}
		\subsubsection{Android SDK}
		\subsubsection{retrofit}