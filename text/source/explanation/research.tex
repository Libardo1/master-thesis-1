\section{Проучване}

	Със създаване на \ac{RS} се занимава \ac{ML} (отрасъл от \ac{CS}). \ac{RS} са интересна алтернатива на алгоритми за търсене, тъй като те намират обекти, които може да нямат общо с термина за търсене. Тези системи, предоставят списък от препоръки, използвайки \ac{CF} или \ac{CBF}. 
	
	\begin{itemize}
		\item[\ac{CF}] подход, при който се изгражда модел на съществуваща история за потребителя и решения, взети от подобни потребители.
	Този модел се използва за предвиждане на какви нови обекти може да се харесат на потребителя.\cite{Melville}
		\item[\ac{CBF}] използва серии от абстрактни характеристики на обектите, за да препоръча подобни обекти.\cite{Mooney}
	\end{itemize}
	
	Пример\footnote{\url{http://en.wikipedia.org/wiki/Recommender_system}} за използване на двата подхода са 
  \emph{Last.fm}\footnote{\url{http://www.last.fm/}} и \emph{Pandora Radio}\footnote{\url{http://www.pandora.com/}}
  
  \begin{itemize}
		\item Pandora използва характеристиките на песен или артист за да избере радио станция, която предоставя песни с подобни характеристики. Мнението на потребителя се използва за определеня на тежест на различните характеристики на радио стацията. Pandora е пример за \ac{CBF}
		\item Last.fm създава виртуална радио станция на базата на историята на потребителя (какви песни и групи е слушал). Тя се сравнява с историята и предпочитанията на други потребители като по този начин, системата предоставя нови песни. Last.fm е пример за \ac{CF}
  \end{itemize}

	Двата подхода имат слаби и силни страни. \ac{CF} се нуждае от голямо количество от информация за да направи качествени препоръки. \ac{CBF} има нужда от малко данни за да започне работата си, но е лимитиран до първоначално подадени данни.
	
	\subsection{Преглед на \ac{CF}}
	
		\ac{CF} е подход за създаване на \ac{RS}, който се налага в практическите имплементации на подобни системи. Основно предимство
		на метода е, че не разчита на анализиране на съдържание от машина и поради тази причина има възможност за точно препоръчване на
		сложни обекти (напр. филми), като не е необходимо разбиране на самия обект.
		
		Използват се различни алгоритми за оценяване сходността на два обекта в \ac{RS}. Два от най-използваните са \ac{KNN} и \ac{PC}.
		
		\begin{itemize}
			\item[\ac{KNN}] е един от най-простите от всички \ac{ML} алгоритми. Обектът е класифициран спрямо мажоратирен вот от неговите съседи, като той е поставен между най-близкия клас от възможните К.
			\item[\ac{PC}] е мярка за линейна свързаност между две променливи в интервала [-1:+1].
		\end{itemize}
		
	\subsubsection{Проблеми с \ac{CF}}
		
		\ac{CF} има три основни проблема: "студен старт", скалируемост и рядкост
		
		\begin{itemize}
			\item студен старт системата изисква много информация за да предостави добри препоръки. В началото на нейното съществуване, обикновено, такава не е налична.
			\item скалируемост много \ac{RS} оперират върху милиони потребители и обекти. Нужна е голяма изчислителна мощ за да се предоставят точни препоръки
			\item рядкост броят на обектите, обикновено, е в пъти по-голям от този на потребителите. Дори и най-активните потребители оценяват само малко подмножество от обектите. Това води до малък брой оценки за отделните обекти
		\end{itemize}
		
	\subsection{Съществуващи \ac{RS}}
	
		\subsubsection{Amazon}
		
			Amazon\footnote{\url{http://www.amazon.com/}} използва \ac{RS} за препоръчване на нови продукти на потребителите си. Предоставя такива, които смята, че ще са интересни за тях. Използва се \ac{CBF}.
			
			Интересно за Amazon e, че има над 29 милиона потребителя и няколко милионен каталог от продукти. Размер данни, който затруднява голяма част от алгоритмите за намиране на препоръки. Разработчиците на Amazon се справят с този проблем като използват офлайн създаване на таблици със сходни продукти. В резултат на това, алгоритъма се грижи само да извлече данни от таблицата, когато те са нужни.
	
		\subsubsection{Youtube}
		
			Youtube\footnote{\url{http://www.youtube.com/}} използва \ac{RS} за да предостави на потребителите си видео записи на интересни за тях теми. Системата се опитва да максимизира броя видеота, които потребителя гледа и времето което прекарва в сайта. Използва се хибридна версия между \ac{CF} и \ac{CBF}.
			
			Лимитиращи фактори са взети под предвид. Системата предоставя само определен брой видео клипове от същия потребител. Използват се уникални за потребителя предпочитания, неговата история, брой изгледани видеота и време по което са гледани за да се увеличи възможността за добра препоръка.
	
	\subsection{Android и пазарът за мобилни приложения}
	
		Android\footnote{\url{http://www.android.com}} е мобилна \ac{OS} базирана на Линукс, създадена предимно за мобилни устройства, като таблети и телефони. Тя се разработва от Google, които през 2005 я закупуват от малка компания\cite{Elgin}. С появата на Android се основава и Open Handset Alliance\footnote{\url{http://www.openhandsetalliance.com/}}, организация която се грижи за развитието на мобилните технологии.
	
		\subsubsection{Нарастващ брой приложения}
	
			Android нараства бързо след 2009 г., когато представлява само 2.8\%\footnote{\url{http://appleinsider.com/articles/09/08/21/canalys_iphone_outsold_all_windows_mobile_phones_in_q2_2009.html}} от пазара за мобилни устройства. В края на 2010 г. представя 33\%\footnote{\url{http://www.canalys.com/newsroom/google\%E2\%80\%99s-android-becomes-world\%E2\%80\%99s-leading-smart-phone-platform}} от него. В края на 2012 г., Android е на челно място с 75\%\cite{IDC}. Активирани са над 900 милиона Android устройства\cite{Google}.
			
			Броят приложения нараства заедно с популярността на Android. Техният брой за Февруари 2013 г. е 800,000 \footnote{\url{www.rssphone.com/google-play-store-800000-apps-and-overtake-apple-appstore/}}. Броят сваляния е около 40 милиарда.
			
			Нарастването на Android означава, че при създаване на \ac{RS} посредством \ac{CF} ще има достатъчно информация за даване на добри препоръки. \emph{Google}, разбира се, вече вграждат подобни методи в пазара на Android. Те са лимитирани до данни, които се предлагат само от техни продукти. Приятелите във \emph{Facebook}\footnote{\url{http://www.facebook.com}}, също биха допринесли за предоставяне на по-добри препоръки при създаване на подобна система. Това не се използва от \emph{Google}.