\subsubsection{Проблеми по време на разработка}

	Системата е разработена под \emph{Ubuntu \ac{OS}}, която предоставя мощни инструменти за създаване на подобни проекти. Мениджърът на пакети (\emph{aptitude}), понякога предоставя стари версии, които не предоставят пълната функционалност на по-новите такива.
	
	\emph{Python} е един такъв пример. Стандартната версия, която се предоставя е 2.7.x. Тя е добре поддържана, но липсват много от възможностите на 3.3.x. 
	
	Проблемът с \emph{Python} се състой в това, че много модули не поддържат последната версия на езика. Пример за такъв е \emph{Scrapy}. Налага се използването на виртуални среди и използването на различни версии за езика, за различни части от системата. Още повече, синтаксиса на езика в някои от версиите се различава.
	
	Официално поддържаните версии на повечето хостинг доставчици все още са в диапазона 2.5.x - 2.7.x. Разработчиците на библиотеки, искат да използват кода си в продукция и насочват усилията си към поддръжка на тези версии. Така, новите възможности на езика не се предоставят на разработчиците, ползващи тези библиотеки.
	
	За създаване на сървърният модул се използват множество малки библиотеки, които предоставят малко документация и липсва стандартен подход за комбинирането им с други такива. \emph{Behave} предоставя инфраструктурата за създаване на тестове, но не предоставя вградена библиотека на самото писане на тестове, което прави първоначалното настройване, трудно.
	
	Клиентската част предоставя няколко проблеми, един от които е фрагментацията. \emph{Android} предоставя възможност за инсталация върху множество различни устройства вариращи от телефони до телевизори, хладилници и други. Много от тях нямат очакваната функционалност, поради различния хардуер, който използват. Версиите на \ac{OS} варират\footnote{\url{http://www.androidbg.com/android-statistics-for-may-2013}}, като по-старата \emph{Gingerbread}, все още доминира над по-новите такива.
	
	Това поставя поне два проблема пред разработчиците на системата:
	
	\begin{itemize}
		\item Кои устройства ще се поддържат?
		\item Кои версии ще се поддържат?
	\end{itemize}
	
	Основен проблем за различните устройства са и различните гъстоти на пиксели, резолюции и големина на екраните. Изграждат се поне два различни изгледа са таблети и телефони. Системата поддържа устройства с версия на \emph{Android}, поне Gingerbread, което налага неизползването на някои по-нови функционалности или само частично такова.
	
	Необходимо е избиране на библиотеки за клиентската част, които са ефективни и бързи, за да направят усещането за приложението приятно и удобно за потребителя. Въпреки големият брой вградени библиотеки и сравнително утвърдилата се платформа, все още липсват модули за лесна комуникация с \emph{REST} услуги.